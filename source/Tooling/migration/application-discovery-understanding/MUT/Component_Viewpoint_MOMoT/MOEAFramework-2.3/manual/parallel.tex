% Copyright 2011-2013 David Hadka.  All Rights Reserved.
%
% This file is part of the MOEA Framework User Manual.
%
% Permission is granted to copy, distribute and/or modify this document under
% the terms of the GNU Free Documentation License, Version 1.3 or any later
% version published by the Free Software Foundation; with the Invariant Section
% being the section entitled "Preface", no Front-Cover Texts, and no Back-Cover
% Texts.  A copy of the license is included in the section entitled "GNU Free
% Documentation License".

\chapter{Parallelization}
\label{chpt:parallelization}
When we first introduced the \java{Executor} class in \chptref{chpt:executor}, we demonstrated the \java{distributeOnAllCores()} method as a way to automatically and seamlessly distribute the evaluation across all cores in your local computer.  This section shows how to expand this simple distributed computing methods to large-scale cloud and high-performance computing systems.

\section{JPPF}
\begin{wrapfigure}{l}{4cm}
  \includegraphics[width=4cm]{jppf.png}
\end{wrapfigure}
JPPF is a Java framework for parallel processing.  JPPF is licensed under the Apache license, version 2, which is a free and open-source license.  This section walks demonstrates distributing the problem evaluations across multiple machines using JPPF.

To begin, first download the Server/Driver, Node, and Application Template distributions from \webpage{http://www.jppf.org/}.  

JPPF introduces an overhead of approximately 0.00068 seconds on DTLZ2, which consists of 11 real-valued decision variables and two objectives.  This was run locally on a single machine, so running across a network will incur additional overhead.

\begin{lstlisting}[language=Java]
JPPFClient jppfClient = null;
JPPFExecutorService jppfExecutor = null;

try {
	jppfClient = new JPPFClient();
	jppfExecutor = new JPPFExecutorService(jppfClient);
		
	// setting the batch size is important, as JPPF will only
	// run one job at a time from a client; the batch size lets
	// us group multiple evaluations (tasks) into a single job
	jppfExecutor.setBatchSize(100);
	jppfExecutor.setBatchTimeout(1000);
	
	NondominatedPopulation result = new Executor()
			.withProblemClass(GAA.class)
			.withAlgorithm("NSGAII")
			.withMaxEvaluations(10000)
			.distributeWith(jppfExecutor)
			.run();
} catch(Exception e) {
	e.printStackTrace();
} finally {
	if (jppfExecutor != null) {
		jppfExecutor.shutdown();
	}
		
	if (jppfClient != null) {
		jppfClient.close();
	}
}
\end{lstlisting}


Serial: 1664
4 Core Local: 549
4 Core JPPF: 552 seconds


Speedup = Ts / Tp
        = 3.01